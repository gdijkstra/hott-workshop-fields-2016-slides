\documentclass{beamer}
\usepackage{etex}

\usetheme{Madrid}
\usecolortheme{dolphin}
\setbeamertemplate{navigation symbols}{} 

%include slides.fmt

\usepackage{hyperref}
\usepackage{cleveref}
\usepackage{xspace}
\usepackage{todonotes}
\usepackage{amsmath}
\usepackage{amsthm}
\usepackage{amssymb}
\usepackage{stmaryrd}
\usepackage{url}
\usepackage{mathrsfs}
\usepackage[pdftex, all]{xy}

\newcommand{\todoi}[1]{\todo[inline]{#1}}

\newcommand{\ie}{i.e.\xspace}
\newcommand{\eg}{e.g.\xspace}

\newcommand{\uip}{uniqueness of identity proofs\xspace}
\newcommand{\Uip}{Uniqueness of identity proofs\xspace}
\newcommand{\hott}{homotopy type theory\xspace}
\newcommand{\Hott}{Homotopy type theory\xspace}
\newcommand{\mltt}{Martin-L\"of's type theory\xspace}

\newcommand{\hit}{higher inductive type\xspace}
\newcommand{\Hit}{Higher inductive type\xspace}
\newcommand{\hits}{higher inductive types\xspace}
\newcommand{\Hits}{Higher inductive types\xspace}
\newcommand{\wtypes}{$W$-types\xspace}
\newcommand{\wtype}{$W$-type\xspace}
\newcommand{\oit}{ordinary inductive type\xspace}
\newcommand{\Oit}{Ordinary inductive type\xspace}
\newcommand{\oits}{ordinary inductive types\xspace}
\newcommand{\Oits}{Ordinary inductive types\xspace}

\newcommand{\zerohit}{$0$-HIT\xspace}
\newcommand{\zerohits}{$0$-HITs\xspace}
\newcommand{\onehit}{$1$-HIT\xspace}
\newcommand{\onehits}{$1$-HITs\xspace}
\newcommand{\twohit}{$2$-HIT\xspace}
\newcommand{\twohits}{$2$-HITs\xspace}
\newcommand{\nhit}[1]{$#1$-HIT\xspace}
\newcommand{\nhits}[1]{$#1$-HITs\xspace}

\newcommand{\ronehit}{restricted \onehit}
\newcommand{\ronehits}{restricted \onehits}
\newcommand{\Ronehit}{Restricted \onehit}
\newcommand{\Ronehits}{Restricted \onehits}

\newcommand{\zeroconstructor}{$0$-constructor\xspace}
\newcommand{\zeroconstructors}{$0$-constructors\xspace}
\newcommand{\oneconstructor}{$1$-constructor\xspace}
\newcommand{\oneconstructors}{$1$-constructors\xspace}
\newcommand{\twoconstructor}{$2$-constructor\xspace}
\newcommand{\twoconstructors}{$2$-constructors\xspace}
\newcommand{\nconstructor}[1]{$#1$-constructor\xspace}
\newcommand{\nconstructors}[1]{$#1$-constructors\xspace}

\newcommand{\data}{{\bf data}}
\newcommand{\where}{{\bf where}}

\newcommand{\Type}{\mathsf{Type}}

\newcommand{\Nat}{\mathbb{N}}
\newcommand{\zero}{\mathsf{zero}}
\newcommand{\suc}{\mathsf{succ}}

\newcommand{\constr}[1]{#1}
\newcommand{\alg}[1]{\mathcal{#1}}
\newcommand{\algs}[1]{\mbox{alg}_{#1}}
\newcommand{\algcat}[1]{#1\mbox{-alg}}
\newcommand{\Algcat}[1]{#1\mbox{-Alg}}
\newcommand{\cat}[1]{\mathcal{#1}}
\newcommand{\Cat}{\mathsf{Cat}}
\newcommand{\Id}[1]{\mathsf{Id}_{#1}}
\newcommand{\id}[1]{\mathsf{id}_{#1}}
\DeclareMathOperator*{\ido}{id_0}
\DeclareMathOperator*{\idi}{id_1}

\newcommand{\ddefeq}{:\equiv}
\newcommand{\defeq}{\equiv}

\newcommand{\cont}[2]{#1 \lhd #2}
\newcommand{\cext}[2]{\llbracket \cont{#1}{#2} \rrbracket}

\newcommand{\C}{\cat{C}}
\newcommand{\D}{\cat{D}}
\newcommand{\assoc}{\mathsf{assoc}}
\newcommand{\map}{\mathsf{map}}
\newcommand{\ap}{\mathsf{ap}}
\newcommand{\apply}{\mathsf{apply}}
\newcommand{\apd}{\mathsf{apd}}
\newcommand{\rec}{\mathsf{rec}}
\newcommand{\ind}{\mathsf{ind}}
\newcommand{\inn}{\mathsf{in}}
\newcommand{\funext}{\mathsf{funext}}
\newcommand{\Spec}{\mathsf{Spec}}
\newcommand{\Alg}{\mathsf{Alg}}
\newcommand{\Constr}{\mathsf{Constr}}
\newcommand{\extend}{\mathsf{extend}}

\newcommand{\base}{\mathsf{base}}
\newcommand{\loopcstr}{\mathsf{loop}}
\newcommand{\Eq}{\mathsf{Eq}}
\newcommand{\eq}{\mathsf{eq}}
\newcommand{\catcat}{\mathsf{Cat}}
\newcommand{\trunc}{\mathsf{trunc}}

\newtheorem{prop}{Proposition}[section]
\newtheorem{thm}[prop]{Theorem}
\newtheorem{lem}[prop]{Lemma}
\newtheorem{cor}[prop]{Corollary}
\newtheorem{conjecture}[prop]{Conjecture}
\theoremstyle{definition}
\newtheorem{defn}[prop]{Definition}
\theoremstyle{remark}

\crefname{prop}{proposition}{propositions}
\crefname{thm}{theorem}{theorems}
\crefname{lem}{lemma}{lemmas}
\crefname{cor}{corollary}{corollaries}
\crefname{defn}{definition}{definitions}
\crefname{equation}{\!}{\!}

\DeclareMathOperator{\isinitial}{\mathsf{is-initial}}
\DeclareMathOperator{\iscontr}{\mathsf{is-contr}}
\DeclareMathOperator{\total}{\mathsf{total}}
\DeclareMathOperator{\proj}{\mathsf{proj}}
\DeclareMathOperator{\preimage}{\mathsf{preimage}}
\DeclareMathOperator{\refl}{\mathsf{refl}}
\DeclareMathOperator{\lift}{\mathsf{lift}}
\DeclareMathOperator{\boolnot}{\mathsf{not}}
\DeclareMathOperator{\Bool}{\mathsf{Bool}}
\DeclareMathOperator{\quot}{\mathsf{quot}}
\DeclareMathOperator{\uipc}{\mathsf{uip}}
\DeclareMathOperator{\Tree}{\mathsf{Tree}}
\DeclareMathOperator{\Pretree}{\mathsf{Pretree}}
\DeclareMathOperator{\coe}{\mathsf{coe}}
\DeclareMathOperator{\Ctx}{\mathsf{Ctx}}
\DeclareMathOperator{\Ty}{\mathsf{Ty}}
\DeclareMathOperator{\snoc}{\mathsf{snoc}}
\DeclareMathOperator{\uncurry}{\mathsf{uncurry}}
\DeclareMathOperator{\leftid}{\mathsf{left-id}}
\DeclareMathOperator{\indnat}{\mathsf{ind-nat}}
\DeclareMathOperator{\indnatzero}{\mathsf{ind-nat-zero}}
\DeclareMathOperator{\indnatsucc}{\mathsf{ind-nat-succ}}

\newcommand{\Fam}{\mathsf{Fam}}
\newcommand{\famover}[2]{\Fam_{#1}\ #2}
\newcommand{\fib}{\mathsf{Fib}}
\newcommand{\fibover}[2]{\fib_{#1}\ #2}
\newcommand{\alghom}[1]{\mathcal{#1}}
\newcommand{\algfam}[1]{\mathcal{#1}}
\newcommand{\algfib}[1]{\mathcal{#1}}
\newcommand{\algsect}[1]{\mathcal{#1}}
\newcommand{\sectionof}[2]{\mathsf{Sect}_{#1}\ #2}
\newcommand{\actionsection}[1]{\bar{#1}}
\newcommand{\dephom}[2]{\mathsf{DepHom}_{#1}\ #2}

\newcommand{\inverseImage}[1]{\mathsf{fib}_{#1}}

% Path composition from The Book.
\newcommand{\ct}{%
  \mathchoice{\mathbin{\raisebox{0.5ex}{$\displaystyle\centerdot$}}}%
             {\mathbin{\raisebox{0.5ex}{$\centerdot$}}}%
             {\mathbin{\raisebox{0.25ex}{$\scriptstyle\,\centerdot\,$}}}%
             {\mathbin{\raisebox{0.1ex}{$\scriptscriptstyle\,\centerdot\,$}}}
}

\newcommand{\To}{\Rightarrow}
\newcommand{\unitty}{\top}
\newcommand{\toruswrong}{T^2\mbox{-}\mathsf{wrong}}
\newcommand{\Fid}{F\mbox{-}\mathsf{id}}
\newcommand{\Fcomp}{F\mbox{-}\circ}
\newcommand{\Gid}{G\mbox{-}\mathsf{id}}
\newcommand{\Gcomp}{G\mbox{-}\circ}
\newcommand{\Gtimes}{G\mbox{-}\times}
\newcommand{\Geq}{G\mbox{-}\mathsf{eq}}
\newcommand{\reasontext}[1]{\{\ \textsf{#1}\ \}}
\newcommand{\reasonterm}[1]{\{\ #1\ \}}
\newcommand{\initialfield}{\mathcal{F}}
\newcommand{\omegacat}{$(\infty,1)$-category\xspace}
\newcommand{\omegacats}{$(\infty,1)$-categories\xspace}
\newcommand{\obj}{\mathsf{obj}}
\newcommand{\Hom}{\mathsf{hom}}
\newcommand{\true}{\mathsf{true}}
\newcommand{\false}{\mathsf{false}}
\newcommand{\toequiv}{\overset{\sim}\to}


\title[Induction and initiality for \onehits]{Induction and homotopy initiality for a class of \onehits}

\author[Gabe Dijkstra]{
  Gabe Dijkstra
}

\institute[University of Nottingham]{
  University of Nottingham
 }

\date[May 20th, 2016]{May 20th, 2016 \\ \vspace{1cm} \small{j.w.w.\ Thorsten Altenkirch, Paolo Capriotti and Fredrik Nordvall Forsberg}}

\begin{document}


\begin{frame}
\maketitle
\end{frame}

\begin{frame}
  \frametitle{Aims}
  Ultimately, we want to have a formal definition of \hits in type
  theory, such that we can do things like:

  \begin{itemize}
  \item show that initiality and induction coincide
  \item prove correctness of hub-spokes construction
  \item show that we get all \hits from some primitive \hit
  \end{itemize}

\end{frame}

\begin{frame}
  \frametitle{Aim of this talk}
  In this talk, we will show:
  
  \begin{itemize}
  \item how to show that induction and initiality coincide generally
  \item how to specify \onehits in type theory
  \item what coherence issues show up when we do so
  \item how to deal with these by restricting ourselves to a smaller class of \onehits
  \item how to derive an induction principle for these \onehits
  \end{itemize}

\end{frame}

\begin{frame}
  \frametitle{Category theory in type theory}
  \begin{itemize}
  \item We want to do category theory in type theory
  \item We do not want to truncate anything: we work with
    hom-\emph{types}, not hom-\emph{sets}
  \item We also do not want to talk about \omegacats
  \item Instead, we deal with coherence \emph{lazily}: we keep track
    of how much structure and coherence we need from our categories
    and functors and provide exactly that
  \end{itemize}
\end{frame}

\begin{frame}
  \frametitle{Initiality and induction}
  Given an endofunctor $F : \Type \to \Type$, we can think of the
  \emph{inductive type} $T$ generated by $F$ in two ways:

  \begin{itemize}
  \item $T$ with its constructor $c : FT \to T$ form an initial object
    in the category $\algcat{F}$ of $F$-algebras
  \item $T$ satisfies an induction principle for $F$
  \end{itemize}
\end{frame}

\begin{frame}
  \frametitle{Initiality}
  Define category $\algcat{F}$:
  \begin{itemize}
  \item objects:
    \begin{align*}
      &| \algcat{F} | : \Type \\
      &| \algcat{F} | \ddefeq (X : \Type) \times (\theta : FX \to X) 
    \end{align*}

  \item morphisms: 
    \begin{align*}
      &\algcat{F}(\_,\_) : | \algcat{F} | \to | \algcat{F} | \to \Type \\
      &\algcat{F}((X,\theta),(Y,\rho)) \ddefeq (f : X \to Y) \times (f_0 : f \circ \theta = \rho \circ Ff)
    \end{align*}
  \end{itemize}

  Note that the computation rule holds up to propositional equality.

  \begin{block}{Definition}
    An object $X$ of category $\C$ is \emph{(homotopy) initial} if
    we have: 
    $$(Y : | \C |) \to \iscontr \ \C(X,Y)$$
  \end{block}

  Inductive type $T$ is the carrier of the initial object of
  $\algcat{F}$ with its constructor $c : FT \to T$ as its algebra
  structure.
\end{frame}

\begin{frame}
  \frametitle{Induction principle}
  $T : \Type$ with constructor $c : FT \to T$ satisfies the
  induction principle if for any \emph{algebra family}:

  \begin{itemize}
  \item $P : T \to \Type$
  \item $m : (x : FT) \times \Box_F\ P\ x \to P\ (\theta\ x)$
  \end{itemize}

  we get a \emph{dependent algebra morphism}:

  \begin{itemize}
  \item $s : (x : T) \to P\ x$
  \item $s_0 : (x : FT) \to s\ (\theta\ x) = m\ x\ (\actionsection{F}\ s\ x)$
  \end{itemize}

  \begin{block}{Notation}
    \begin{itemize}
    \item $\Box_F\ P : FT \to \Type$ lifts the family $P$ on $T$ to
      $FT$
    \item
      $\actionsection{F} : ((x : T) \to P\ x) \to (x : FT) \to \Box_F\
      P\ x$ lifts dependent functions on $P$ to $\Box_F\ P$
    \end{itemize}
  \end{block}
\end{frame}

\begin{frame}
  \frametitle{Initiality versus induction in type theory}
  \begin{itemize}
  \item For \oits initiality and induction have been shown to be
    equivalent (Sojakova et al., 2012)
  \item For initiality we only need objects and morphisms
  \item For the induction principle we need more structure...
  \end{itemize}
\end{frame}

\begin{frame}
  \frametitle{Induction -- categorically}
  Instead of proving:
  \begin{center}
    initiality $\iff$ induction
  \end{center}
  directly, we introduce an intermediate concept:
  \begin{center}
    initiality $\iff$ \emph{section} induction $\iff$ induction
  \end{center}
  \hfill

  \begin{block}{Definition}
    An object $X : | \C |$ satisfies the \emph{section induction
      principle} if for every $Y : | \C |$ any $p : \C(Y,X)$ has a
    section $s : \C(X,Y)$
  \end{block}
\end{frame}

\begin{frame}
  \frametitle{Initiality implies induction -- categorically}
  
  Suppose $X : | \C |$ initial, then for any $Y$ with $p : \C(Y,X)$ we
  have a unique $s : \C(X,Y)$ which is a section of $p$:
  $$
  \xymatrix{
    \uncover<2->{X} \only<2->{\ar[r]^{s}} \only<3->{\ar[dr]_{\Id{X}}} &Y \ar[d]^{p} \\
    &X
  }
  $$

\end{frame}

\begin{frame}
  \frametitle{Induction implies initiality -- categorically}

  Suppose we have $X : | \C |$ that satisfies the section induction
  principle. Assuming $\C$ has products, then for any $Y$ we have the
  projection:
  $$
  \xymatrix{
    X \times Y \ar[r]^-{\pi_1} &X
  }
  $$
  which has a section $s$. Define $f : \C(X,Y)$ as the composite:
  $$
  \xymatrix{
    X \ar[r]^-{s} &X \times Y \ar[r]^-{\pi_2} &Y
  }
  $$
  We have to show that any other $g : \C(X,Y)$ is equal to this
  $f$. Taking the equaliser of $f$ with any $g$, we get:

  $$
  \xymatrix{ 
    &\uncover<2->{E} \only<2->{\ar[r]^{e}} &X \ar@<-.5ex>[r]_-{g} \ar@<.5ex>[r]^-{f} &Y \\
    &\uncover<3->{X} \only<3->{\ar[u]^{s} \ar[ur]_{\Id{X}}}
  }
  $$
\end{frame}

\begin{frame}
  \frametitle{Category structure needed}

  To formalise the previous arguments, we need the following:

  \begin{itemize}
  \item For initiality: objects and morphisms
  \item For sections:
    \begin{itemize}
    \item Composition
    \item Identity morphisms
    \end{itemize}
  \item To show that initiality and section principle coincide:
    \begin{itemize}
    \item Associativity
    \item Identity laws
    \item Products
    \item Equalisers
    \end{itemize}
  \end{itemize}

\end{frame}

\begin{frame}
  \frametitle{Category structure needed - \oits}
  
  Given $F : \Type \to \Type$ as a \emph{container}, we can define $\algcat{F}$ as:
  \begin{itemize}
  \item objects:
    \begin{align*}
      &| \algcat{F} | : \Type \\
      &| \algcat{F} | \ddefeq (X : \Type) \times (\theta : FX \to X) 
    \end{align*}

  \item morphisms: 
    \begin{align*}
      &\algcat{F}(\_,\_) : | \algcat{F} | \to | \algcat{F} | \to \Type \\
      &\algcat{F}((X,\theta),(Y,\rho)) \ddefeq (f : X \to Y) \times (f_0 : f \circ \theta = \rho \circ Ff)
    \end{align*}
  \end{itemize}

  \begin{itemize}
  \item With some effort, we can define all the categorical structure needed
  \item However, categorical laws are not satisfied strictly
  \end{itemize}

\end{frame}

\begin{frame}
  \frametitle{\Hits versus \oits}
Ordinary inductive type $T$ with constructors
\begin{itemize}
  \item $c_0 : F_0 T \to T$
  \item $\vdots$
  \item $c_k : F_k T \to T$
\end{itemize}

where every $F_i : \Type \to \Type$ is a (strictly positive) functor.
\end{frame}

\begin{frame}[noframenumbering]
  \frametitle{\Hits versus \oits}
Ordinary inductive type $T$ with constructor:
\begin{itemize}
  \item $c  : F_0 T + \hdots + F_k T \to T$
\end{itemize}

where every $F_i : \Type \to \Type$ is a (strictly positive) functor.
\end{frame}

\begin{frame}[noframenumbering]
  \frametitle{\Hits versus \oits}
Ordinary inductive type type $T$ with constructor:
\begin{itemize}
  \item $c  : F T \to T$
\end{itemize}

where $F : \Type \to \Type$ is a (strictly positive) functor.
\end{frame}

\begin{frame}
  \frametitle{\Hits versus \oits}
  Higher inductive types, \eg the circle $S^1$ has constructors:

  \begin{itemize}
    \item $\base  : S^1$
    \item $\loopcstr  : \base =_{S^1} \base$
  \end{itemize}

  Differences from an \oit:

  \begin{itemize}
  \item Dependencies on previous constructors
  \item \emph{Higher} constructors: target of constructors not always
    $T$, but can also be an iterated path space of $T$.
  \end{itemize}

  Single functor $\Type \to \Type$ no longer suffices
\end{frame}

\begin{frame}
  \frametitle{Specifying \onehits}
  A specification of a \onehit is either:
  \begin{itemize}
  \item empty (denoted $\epsilon$)
  \item a specification $s$ extended with a \zeroconstructor on $\algs{s}$,
  \item or a specification $s$ extended with a \oneconstructor on $\algs{s}$
  \end{itemize}

  Defined mutually with the type of specifications is the function
  that maps a specification to its category of algebras:
  $$
   \algs{} : \Spec \to \Cat
  $$
  
  For the empty specification $\algs{\epsilon} \ddefeq \Type$ 

  We also define mutually the underlying functor $U_s$ which gives us
  the underlying type of the algebra.
\end{frame}

\begin{frame}
  \frametitle{Specifying \onehits\ -- category of algebras -- \zeroconstructors}
  
  Given $s : \Spec$, a \zeroconstructor is given by a functor:
  $$
  F : \algs{s} \to \Type
  $$
  and the category $\algs{(s,F)}$ is defined as having:
  \begin{itemize}
  \item objects: 
    \begin{itemize}
    \item $X : | \algs{s} |$
    \item $\theta : FX \to U_s X$
    \end{itemize}
  \item morphisms $(X,\theta) \to (Y,\rho)$:
    \begin{itemize}
    \item $f : \algs{s}(X,Y)$
    \item $f_0 : U_s f \circ \theta = \rho \circ F f$
    \end{itemize}
  \end{itemize}
\end{frame}

\begin{frame}
  \frametitle{Specifying \onehits\ -- category of algebras -- \oneconstructors}
  
  Given $s : \Spec$, a \oneconstructor is given by specifying its arguments:
  $$
  F : \algs{s} \to \Type
  $$
  and the endpoints of the path, as natural transformations:
  $$
  l, r : F \to U_s
  $$

  The category of algebras is then:
  \begin{itemize}
  \item objects:
    \begin{itemize}
    \item $X : | \algs{s} |$
    \item $\theta : l_X =_{U_s X} r_X$
    \end{itemize}
  \item morphisms $(X,\theta) \to (Y,\rho)$:
    \begin{itemize}
    \item $f : \algs{s}(X,Y)$
    \item $f_0 : \xymatrix{
        Uf \circ l_X
        \ar@{-}[r]^{Uf \circ \theta}
        \ar@{-}[d]_{\alpha_f}
        &Uf \circ r_X
        \ar@{-}[d]^{\beta_f}
        \\
        l_Y \circ Ff 
        \ar@{-}[r]_{\rho \circ Ff}
        &r_Y \circ Ff
      }$
    \end{itemize}
  \end{itemize}
\end{frame}

\begin{frame}
  \frametitle{Functors and natural transformations in type theory}

  \begin{itemize}
  \item Functors $F : \algs{s} \to \Type$ need to be \emph{strictly
    positive}
  \item Strictly positive functors $\Type \to \Type$: \emph{containers}
  \item For a given $\C : Cat$, strictly positive functors
    $\C \to \Type$: \emph{$\C$-containers}
  \item Natural transformations between strictly positive functors: \emph{container morphisms}
  \item Forgetful functors $U_s : \algs{s} \to \Type$ are containers if they have a left adjoint
  \end{itemize}
\end{frame}

\begin{frame}{Containers on $\Type$}
  Strictly positive functors $\Type \to \Type$: containers

  \begin{itemize}
  \item Shapes $S : \Type$
  \item Positions $T : S \to \Type$
  \end{itemize}

  \begin{align*}
  &\llbracket S \lhd P \rrbracket_0 : \Type \to \Type \\
  &\llbracket S \lhd P \rrbracket_0\ X \ddefeq (s : S) \times (P\ s \to X) \\
    \\
  &\llbracket S \lhd P \rrbracket_1 : (X \to Y) \to \llbracket S \lhd P \rrbracket_0\ X \to \llbracket S \lhd P \rrbracket_0\ Y \\
  &\llbracket S \lhd P \rrbracket_1\ f\ (s , t) \ddefeq (s , f \circ t)
  \end{align*}

  The functor laws follow from the categorical laws of $\Type$, \ie
  they are satisfied \emph{strictly}
\end{frame}

\begin{frame}{$\C$-containers}
  Strictly positive functors $\C \to \Type$: $\C$-containers (or \emph{familially representable})

  \begin{itemize}
  \item Shapes $S : \Type$
  \item Positions $T : S \to | \C | $
  \end{itemize}
  %
  with
  % 
  \begin{align*}
  &\llbracket S \lhd P \rrbracket_0 : \C \to \Type \\
  &\llbracket S \lhd P \rrbracket_0\ X \ddefeq (s : S) \times \C(P\ s, X) \\
    \\
  &\llbracket S \lhd P \rrbracket_1 : \C(X , Y) \to \llbracket S \lhd P \rrbracket_0\ X \to \llbracket S \lhd P \rrbracket_0\ Y \\
  &\llbracket S \lhd P \rrbracket_1\ f\ (s , t) \ddefeq (s , f \circ t)
  \end{align*}

  The functor laws follow from the categorical laws of $\C$, \ie
  associativity and identity laws
\end{frame}

\begin{frame}{$\C$-container morphisms}
  Natural transformations between containers: \emph{container
      morphisms}:

  For containers $S \lhd P$ and $T \lhd Q$, container morphisms are:
  $$
  (f : S \to T) \times  (g : (a : S) \to \C (Q\ (f\ a), P\ a))
  $$

  with 
  % 
  \begin{align*}
    &\apply\ (f , g) : (X : | \C |) \to 
      \llbracket S \lhd P \rrbracket_0\ X \to 
      \llbracket T \lhd Q \rrbracket_0\ X \\
    &\apply\ (f , g)\ X\ (s , t) \ddefeq (f\ s , t \circ (g\ s))
  \end{align*}

  Naturality follows from associativity of $\C$
\end{frame}

\begin{frame}{\zerohits and coherence}
  Consider a specification with three \zeroconstructors, \ie the
  category of algebras has objects:
  %
  \begin{itemize}
    \item $X : \Type$
    \item $\theta_0 : F_0 X \to X$
    \item $\theta_1 : F_1 (X,\theta_0) \to X$
    \item $\theta_2 : F_2 (X,\theta_0,\theta_1) \to X$
  \end{itemize}
  % 
  Supposing all functors $F_i$ are given as containers, then:
  % 
  \begin{flalign*}
    &&&\ \mbox{identity morphisms in $\algcat{F_2}$}  \\
    &&\Leftarrow &\ \mbox{identity laws of functor $F_2$} \\
    &&\Leftarrow &\ \mbox{identity laws of category $\algcat{F_1}$} \\
    &&\Leftarrow &\ \mbox{coherence of identity laws of $\algcat{F_0}$ and composition law $F_1$} \\
    &&\Leftarrow &\ \mbox{coherence of identity and associativity laws of $\algcat{F_0}$}
  \end{flalign*}

  Coherence issues increase with the amount of constructors, even if
  we only have \zeroconstructors
\end{frame}

\begin{frame}
  \frametitle{\onehits}
  
  We will look at \onehits $T$ with constructors:
  %
  \begin{itemize}
  \item $c_0 : F_0 T \to T$
  \item $c_1 : (x : F_1 T) \to c_0^*\ (l_T\ x) =_T c_0^*\ (r_T\ x)$
  \end{itemize}
  %
  where:
  %
  \begin{itemize}
  \item $F_0, F_1 : \Type \to \Type$ functors given as containers
  \item $F_0^* : \Type \to \Type$ is the free monad of $F_0$, also given as a container
  \item $c_0^* : F_0^* X \to X$ is the algebra $c_0$ lifted to the free monad $F_0^*$
  \item $l, r : F_1 \to F_0^*$ natural transformations given as container morphisms
  \end{itemize}
  
\end{frame}

\begin{frame}[noframenumbering]
  \frametitle{\onehits}
   
   We will look at \onehits $T$ with constructors:
   
   \begin{itemize}
   \item $c_0 : F_0 T \to T$
   \item $c_1 : c_0^* \circ l_T =_T c_0^* \circ r_T$
   \end{itemize}

   where:

   \begin{itemize}
  \item $F_0, F_1 : \Type \to \Type$ functors given as containers
  \item $F_0^* : \Type \to \Type$ is the free monad of $F_0$, also given as a container
  \item $c_0^* : F_0^* X \to X$ is the algebra $c_0$ lifted to the free monad $F_0^*$
  \item $l, r : F_1 \to F_0^*$ natural transformations given as container morphisms
   \end{itemize}

 \end{frame}

 \begin{frame}
   \frametitle{\onehits\ -- algebras}
   
   Given a specification $(F_0,F_1, l, r)$, the category of algebras has:

   \begin{itemize}
   \item objects:
     \begin{itemize}
     \item $X : Type$
     \item $\theta_0 : F_0 X \to X$
     \item $\theta_1 : \theta_0^* \circ l_X = \theta_0^* \circ r_X$
     \end{itemize}
   \item morphisms $(X,\theta_0,\theta_1) \to (Y,\rho_0,\rho_1)$:
     \begin{itemize}
     \item $f : X \to Y$ 
     \item $f_0 : f \circ \theta_0 = \rho_0 \circ F_0 f$
     \item $f_1 : \xymatrix{ f \circ \theta_0^* \circ l_X \ar@{-}[r]^{f
           \circ \theta_1} \ar@{-}[d]_{f_0^* \circ l_X} &f \circ
         \theta_0^* \circ r_X \ar@{-}[d]^{f_0^* \circ r_X}
         \\
         \rho_0^* \circ l_Y \circ F_1 f \ar@{-}[r]_{\rho_1 \circ F_1 f} &\rho_0^* \circ r_Y
         \circ F_1 f }$
     \end{itemize}
   \end{itemize}
 \end{frame}

 \begin{frame}
   \frametitle{\onehits\ -- algebras}

   Defining the category structure is involved:
   
   \begin{itemize}
   \item $F_0, F_1$ and $l, r$ satisfy their respective laws strictly,
   \item $\_^* : | \algcat{F} | \to | \algcat{F^*} |$ is not strictly functorial however
   \item We need to show that its functoriality is coherent with the category structure
   \item There is a lot of path algebra involved
   \item Cubical methods from the \texttt{HoTT-Agda} library make life a bit easier
   \end{itemize}

 \end{frame}

 \begin{frame}
   \frametitle{\onehits\ -- induction principle}  
   We need to define algebra families and dependent algebra morphisms

   We have:
   $$
   (X \to \Type) = (Y : \Type) \times (p : Y \to X)
   $$
   as witnessed by:
   $$
   to\ P \ddefeq (\Sigma X P , \pi_1)
   $$
   $$
   from\ (Y, p) \ddefeq p^{-1}
   $$
   
   Under this equivalence, \emph{sections} correspond to
   \emph{dependent functions}

   \vfill
   We can derive algebra families and dependent algebra families by
   finding similar equivalences, replacing types and functions with
   algebras and algebra morphisms.
 \end{frame}

 \begin{frame}
   \frametitle{\onehits\ -- induction principle -- families}
   Suppose $(X,\theta_0,\theta_1) : | \algs{(F_0,F_1,l,r)} |$, then
   the type of algebra families over $(X,\theta_0,\theta_1)$ is an
   $M : (X \to \Type) \to \Type$ that satisfies satisfies the
   equation:
   % 
   \begin{alignat*}{2}
   (P : X \to \Type) \times M\ P &\ &=\      &(Y : \Type) \\
                                 &\ &\times\ &(\rho : FY \to Y) \\
                                 &\ &\times\ &(p : Y \to X) \\
                                 &\ &\times\ &(p_0 : \hdots) \\
                                 &\ &\times\ &(p_1 : \hdots)
   \end{alignat*}
 \end{frame}

 \begin{frame}
   \frametitle{\onehits\ -- induction principle -- families}
   We can solve the previous equation by applying the witnesses of the
   equivalence $(P : X \to \Type) = (Y : \Type) \times (p : Y \to
   X)$. An algebra family then consists of:
   %
   \begin{align*}
     &&&\ (P : X \to Type) \\
     &&\times&\ (m_0 : (x : F_0 X) \times \Box_{F_0}\ P\ x \to P\ (\theta_0\ x)) \\
     &&\times&\ (m_1 : (x : F_1 X) \times (y : \Box_{F_1}\ P\ x) \\
     &&&\ \to  m_0^*\ (l^d (x , y)) = m_0^*\ (r^d (x , y))\ [ P \downarrow \theta_1\ x ])
   \end{align*}
   % 
   where
   
   \begin{itemize}
   \item $\Box_F$ satisfies $F (\Sigma X P) = \Sigma (FX) (\Box_{F}\ P)$
   \item $m_0^* : (x : F_0^* X) \times \Box_{F_0^*}\ P\ x \to P\ (\theta_0^*\ x)$
   \item $l^d , r^d : (x : F_1 X) \times \Box_{F_1}\ P\ x \to (x : F_0^* X) \times \Box_{F_0^*}\ P\ x$
   \end{itemize}
\end{frame}

 \begin{frame}
   \frametitle{\onehits\ -- induction principle -- dependent morphisms}
   We can also figure out what under this equivalence the sections
   amount to and we arrive at the following, given an algebra family:
   % 
   \begin{itemize}
     \item $P : X \to Type$
     \item $m_0 : (x : F_0 X) \times \Box_{F_0}\ P\ x \to P\ (\theta_0\ x)$
     \item $m_1 : (x : F_1 X) \times (y : \Box_{F_1}\ P\ x)$ $\to  m_0^*\ (l^d (x , y)) = m_0^*\ (r^d (x , y))\ [ P \downarrow \theta_1\ x ]$
   \end{itemize}
   % 
   a dependent algebra morphism over $(P,m_0,m_1)$ consists of:
   % 
   \begin{itemize}
   \item $s : (x : X) \to P\ x$
   \item $s_0 : (x : F_0 X) \to s\ (\theta_0\ x) = m_0\ x\ (\actionsection{F_0}\ s\ x)$
   \item $s_1 : (x : F_1 X) \to \xymatrix{
       s\ (\theta_0^*\ (l\ x)) 
       \ar@{-}[rr]^{s\ (\theta_1\ x)}_{[ P\ \downarrow\ \theta_1\ x ]}
       \ar@{-}[d]_{s_0^*\ (l\ x)}
       &&s\ (\theta_0^*\ (r\ x)) 
       \ar@{-}[d]^{s_0^*\ (r\ x)} \\
       m_0^* (l^d (x , \actionsection{F_1}\ s\ x)) 
       \ar@{-}[rr]^{m_1\ x\ (\actionsection{F_1}\ s\ x)}_{[ P\ \downarrow\ \theta_1\ x ]}
       &&m_0^* (r^d (x , \actionsection{F_1}\ s\ x))} $
   \end{itemize}

 \end{frame}

 \begin{frame}
   \frametitle{Conclusions}
   
   \begin{itemize}
   \item We can
     \begin{itemize}
     \item define \onehits in type theory
     \item define induction for a class of them
     \item show that for this class induction and initiality coincide
     \end{itemize}
   \item Coherence issues increase with the number of constructors
   \item Dealing with these generally requires heavy machinery, \eg \omegacats
   \item Agda formalisation is a work in progress: \url{https://github.com/gdijkstra/homotopy-initiality}
   \end{itemize}

 \end{frame}

\end{document}
