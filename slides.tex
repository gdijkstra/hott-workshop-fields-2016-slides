\documentclass{beamer}
\usepackage{etex}

\usetheme{Madrid}
\usecolortheme{dolphin}
\setbeamertemplate{navigation symbols}{} 

%include slides.fmt

\usepackage{hyperref}
\usepackage{cleveref}
\usepackage{xspace}
\usepackage{todonotes}
\usepackage{amsmath}
\usepackage{amsthm}
\usepackage{amssymb}
\usepackage{stmaryrd}
\usepackage{url}
\usepackage{mathrsfs}
\usepackage[pdftex, all]{xy}

\input{macros.tex}

\title[Induction and initiality for \onehits]{Induction and homotopy initiality for a class of \onehits}

\author[Gabe Dijkstra]{
  Gabe Dijkstra
}

\institute[University of Nottingham]{
  University of Nottingham
 }

\date[May 20th, 2016]{May 20th, 2016 \\ \vspace{1cm} \small{j.w.w.\ Thorsten Altenkirch, Paolo Capriotti and Fredrik Nordvall Forsberg}}

\begin{document}


\begin{frame}
\maketitle
\end{frame}

\begin{frame}
  \frametitle{Goal}
  \begin{itemize}
  \item Ultimately: have a way to specify \hits
  \item For every specification we should be able to generate:
    \begin{itemize}
    \item the category of algebras
    \item introduction rules and induction principle
    \end{itemize}
  \item \emph{Sanity check}: show that initiality and induction coincide
  \item In this talk: initiality and induction coincide for a class of \onehits
  \item We want to do this all in type theory itself
  \end{itemize}
\end{frame}

\begin{frame}
  \frametitle{Category theory in type theory}
  \begin{itemize}
  \item We want to talk about categories in type theory
  \item We do not want to truncate anything: we work with
    hom-\emph{types}, not hom-sets
  \item We also do not want to talk about \omegacats
  \item We will deal with coherence \emph{lazily}: we keep track of
    how much structure and coherence we need from our categories and
    functors and provide exactly that
  \end{itemize}
\end{frame}

\begin{frame}
  \frametitle{Initiality and induction}
  \todoi{Put this on two slides?}
  Given an endofunctor $F : \Type \to \Type$, we can think of the
  \emph{inductive type} $T$ generated by $F$ in two ways:

  \begin{columns}
    \begin{column}[t]{0.5\textwidth}
      \emph{Initiality}

      Define category $\algcat{F}$:
      \begin{itemize}
      \item objects: $(X : \Type) \times (\theta : FX \to X)$
      \item morphisms $(X,\theta) \to (Y,\rho)$:
        $(f : X \to Y) \times (f_0 : f \circ \theta = \rho \circ Ff)$
      \end{itemize}

      $T$ is the carrier of the initial object of $\algcat{F}$ with
      its constructor $c : FT \to T$ its algebra structure.
    \end{column}
    \begin{column}[t]{0.5\textwidth}
      \emph{Induction}

      $T : \Type$ with constructor $c : FT \to T$ satisfies the
      induction principle if for all:

      \begin{itemize}
      \item $P : T \to \Type$
      \item $m : (x : FT) \times \Box_F\ P\ x \to P (\theta\ x)$
      \end{itemize}

      we get:

      \begin{itemize}
      \item $s : (x : T) \to P\ x$
      \item $s_0 : (x : FT)$\\
        $\to s\ (\theta\ x) = m\ x\ (\actionsection{F}\ s\ x)$
      \end{itemize}
    \end{column}
  \end{columns}
\end{frame}

\begin{frame}
  \frametitle{Initiality versus induction in type theory}
  \begin{itemize}
  \item An object $X$ of category $\C$ is \emph{(homotopy) initial} if we have:
    $(Y : | \C |) \to \iscontr(\C(X,Y))$
  \item For \oits initiality and induction are equivalent (Sojakova et
    al.) \todoi{Get proper year citation and stuff}
  \item For initiality we only need objects and morphisms
  \item For induction we need more structure
  \end{itemize}
\end{frame}

\begin{frame}
  \frametitle{Induction categorically}
  Instead of proving: \\
  initiality $\iff$ induction \\
  directly, we instead show: \\
  initiality $\iff$ \emph{section} induction $\iff$ induction \\
  
  An object $X : | \C |$ satisfies the \emph{section induction
    principle} if for every $Y : | \C |$ any $p : \C(Y,X)$ has a
  section $s : \C(X,Y)$.
\end{frame}


\begin{frame}
  \frametitle{Initiality implies induction}
  
  \todoi{Show animation for this}
\end{frame}

\begin{frame}
  \frametitle{Induction implies initiality}
  
  \todoi{Show animation for this}

\end{frame}

\begin{frame}
  \frametitle{Structure needed}
  
  \begin{itemize}
  \item To talk about initiality, defining objects and morphisms suffices
  \item Defining sections requires:
    \begin{itemize}
    \item Composition
    \item Identity morphisms
    \end{itemize}
  \item Showing that initiality and section principle coincide requires:
    \begin{itemize}
    \item Products
    \item Equalisers
    \item Associativity
    \item Identity laws
    \end{itemize}
  \end{itemize}
\end{frame}

\begin{frame}
  \frametitle{\Hits versus \oits}
Ordinary inductive type $T$ with constructors
\begin{itemize}
  \item $c_0 : F_0 T \to T$
  \item $\vdots$
  \item $c_k : F_k T \to T$
\end{itemize}

where every $F_i : \Type \to \Type$ is a (strictly positive) functor.
\end{frame}

\begin{frame}[noframenumbering]
  \frametitle{\Hits versus \oits}
Ordinary inductive type $T$ with constructor:
\begin{itemize}
  \item $c  : F_0 T + \hdots + F_k T \to T$
\end{itemize}

where every $F_i : \Type \to \Type$ is a (strictly positive) functor.
\end{frame}

\begin{frame}[noframenumbering]
  \frametitle{\Hits versus \oits}
Ordinary inductive type type $T$ with constructor:
\begin{itemize}
  \item $c  : F T \to T$
\end{itemize}

where $F : \Type \to \Type$ is a (strictly positive) functor.
\end{frame}

\begin{frame}
  \frametitle{\Hits versus \oits}
  Higher inductive types, \eg the circle $S^1$ has constructors:

  \begin{itemize}
    \item $\base  : S^1$
    \item $\loopcstr  : \base =_{S^1} \base$
  \end{itemize}

  \begin{itemize}
  \item Dependencies on previous constructors
  \item \emph{Higher} constructors: target of constructors not always
    $T$, but can also be an iterated path space of $T$.
  \end{itemize}

  Single functor $\Type \to \Type$ no longer suffices
\end{frame}

\begin{frame}
  \frametitle{General framework}
  Constructors are \emph{dependent dialgebras}, type $T$ with constructors:

  \begin{itemize}
    \item $c_0  : (x : F_0 T) \to G_0 (T, x)$
    \item $c_1  : (x : F_1 (T, c_0)) \to G_1 ((T, c_0), x)$
    \item $\vdots$
    \item $c_k  : (x : F_k (T, c_0, \hdots, c_{k-1}))  \to G_k ((T, c_0, \hdots, c_{k-1}), x)$
  \end{itemize}

%   We will call:

%   \begin{itemize}
%   \item Every |Fi| an \emph{argument} functor
%   \item Every |Gi| a \emph{target} functor
%   \end{itemize}
 \end{frame}


\end{document}
